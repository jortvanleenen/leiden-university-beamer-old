%% example.tex
%% Copyright 2023 J.P. van Leenen
%
% This work may be distributed and/or modified under the
% conditions of the LaTeX Project Public License, either version 1.3
% of this license or (at your option) any later version.
% The latest version of this license is in
%   https://www.latex-project.org/lppl.txt
% and version 1.3c or later is part of all distributions of LaTeX
% version 2008 or later.
%
% This work has the LPPL maintenance status `maintained'.
% 
% The Current Maintainer of this work is J.P. van Leenen.

% The theme's footer has an automatic translation for 'dutch', all other languages will display their footer in English by default. Manual changes can be made in 'beamerouterthemelu.sty'.
\documentclass[english, aspectratio=169]{beamer}

% OPTIONS FOR THEME:
% - footer: if set, can be used to change the layout of the footer. Information displays the author short, institution short and title short. Empty removes all text except for the slide numbering. None removes the footer completely.
%     Default: identical to Leiden University's PPTX, 'Discover the World at Leiden University' in either English or Dutch, see above on the documentclass option.
%     Option(s): information, empty, none
% - style: if set, can be used to change the accent colour to the colour of your faculty of choice.
%     Default: midblueLU colour
%     Option(s): social, science, humanities, archaeology, fgga, medicine, law
% - styleslidenumbering: if set, change the colour of the background of the slide numbering to match the colour of the style option.
%     Default: darkblueLU colour
%     Option(s): <none> 
% - addstandoutfootline: if set, adds the footline to the standout template.
%     Default: footline is removed
%     Option(s): <none>
\usetheme[footer=information]{lu}

\usepackage{appendixnumberbeamer}
\graphicspath{{graphics/}}

% \RequirePackage[style=numeric]{biblatex}
% \addbibresource{literature.bib}

\title{What Makes Leiden University so Great?}
\subtitle{Its People}
\author[surname1, surname2, surname3]{\texorpdfstring{%
    \begin{columns}%
        \begin{column}{.5\paperwidth}%
            \begin{beamercolorbox}[center, sep=1ex]{author}%
                \leftskip=.15\paperwidth%
                \begin{minipage}{.35\paperwidth}
                    \centering
                    \usebeamerfont{author}T. Surname1\ {\footnotesize(1234567)\\ \href{mailto:T.Surname1@leidenuniv.nl}{T.Surname1@leidenuniv.nl}}%
                \end{minipage}
            \end{beamercolorbox}%
        \end{column}%
        \begin{column}{.5\paperwidth}%
            \begin{beamercolorbox}[center, sep=1ex]{author}%
                \begin{minipage}{.35\paperwidth}
                    \centering
                    \usebeamerfont{author}T. Surname2\ {\footnotesize(1234568)\\ \href{mailto:T.Surname2@leidenuniv.nl}{T.Surname2@leidenuniv.nl}}%
                \end{minipage}
                \rightskip=.15\paperwidth%
            \end{beamercolorbox}%
        \end{column}%
    \end{columns}%
    \nointerlineskip%
    \begin{columns}%
    \begin{column}{1\paperwidth}%
        \begin{beamercolorbox}[center, sep=1ex]{author}%
            \usebeamerfont{author}T. Surname3\ {\footnotesize(1234569)\\ \href{mailto:T.Surname3@leidenuniv.nl}{T.Surname3@leidenuniv.nl}}%
        \end{beamercolorbox}%
    \end{column}%
\end{columns}%
}{T.Surname1, T.Surname2, T.Surname3}}
\date[\today]{Occasion\\\today}
% \titlegraphic can be provided to add the logo of an institution or organisation to the title page
% \titlegraphic{%
%     \includegraphics[height=0.3\paperheight]{logo-of-institution-or-organisation.pdf}%
% }


\begin{document}
    \begin{frame}[b]
        \titlepage
    \end{frame}

    \begin{frame}{Some Block and Styling Examples}
        \begin{definition}
            \textbf{ Leiden university} is the \emph{oldest} institution of higher education in the Netherlands.
        \end{definition}

        \begin{alertblock}{Housing}
            If you are considering studying at Leiden University and come from far or abroad, start searching for a place to stay \alert{as soon as possible}!
        \end{alertblock}

        \begin{exampleblock}{Open}
            Everyone is welcome!
        \end{exampleblock}
    \end{frame}

    \begin{frame}[fragile]{\texttt{\textbackslash accent} Usage}
        This theme comes with a custom command called \texttt{\textbackslash accent}, which, by default, behaves exactly like \texttt{\textbackslash alert}. However, additionally, one of the theme's colours can be specified as an option to allow for easy highlighting in colours besides the default one.

        \begin{table}[]
            \centering
            \begin{tabular}{c|c}
                \hline
                \accent{highlighted}             & \verb|\accent{highlighted}|             \\
                \accent[lightblue]{highlighted}  & \verb|\accent[lightblue]{highlighted}|  \\
                \accent[midblue]{highlighted}    & \verb|\accent[midblue]{highlighted}|    \\
                \accent[darkblue]{highlighted}   & \verb|\accent[darkblue]{highlighted}|    \\
                \accent[red]{highlighted}        & \verb|\accent[red]{highlighted}|        \\
                \accent[lightgreen]{highlighted} & \verb|\accent[lightgreen]{highlighted}| \\
                \accent[darkgreen]{highlighted}  & \verb|\accent[darkgreen]{highlighted}|  \\
                \accent[turquoise]{highlighted}  & \verb|accent[turquoise]{highlighted}|   \\
                \accent[violet]{highlighted}     & \verb|\accent[violet]{highlighted}|     \\
                \hline
            \end{tabular}
        \end{table}
    \end{frame}

    \begin{frame}{List Examples}{And a Subtitle Example as Bonus}
        \begin{columns}[onlytextwidth]%
        \begin{column}{.5\textwidth}%
        \textbf{Itemize}
        
            \begin{itemize}
                \item This is the first item.
                \item This is the second item.
                \begin{itemize}
                    \item This is the first subitem.
                    \begin{itemize}
                        \item This is the first subsubitem.
                    \end{itemize}
                \end{itemize}
            \end{itemize}
        \end{column}%
        \begin{column}{.5\textwidth}%
            \textbf{Enumerate}

            \begin{enumerate}
                \item This is the first item.
                \item This is the second item.
                \begin{enumerate}
                    \item This is the first subitem.
                    \begin{enumerate}
                        \item This is the first subsubitem.
                    \end{enumerate}
                \end{enumerate}
            \end{enumerate}
        \end{column}%
    \end{columns}%
    \end{frame}

    \begin{frame}[fragile]{Listings Template}
        Various styles have been provided with this theme for external tools. See the \texttt{styles} folder for external library and application styles. LaTeX-related styles have been added to \texttt{beamerthemelu.sty}. As an example, take the following code listing to which the Listings style \texttt{lu} has been applied.

        \begin{lstlisting}[language=python, style=lu]
print('Hello world!') # An example of Hello World in Python
        \end{lstlisting}
    \end{frame}

    \begin{frame}[c, standout]{Standout}
        This template reverses the colour usage. It could, for example, be used to review or recap information or draw special attention to a particular aspect.
    \end{frame}

    \begin{frame}[closure]{Closure can be used as a template for a final slide. Questions?}
    \end{frame}

    % \appendix
    % \begin{frame}[allowframebreaks]{References}
    %     \printbibliography
    % \end{frame}
\end{document}
