% The theme's footer has an automatic translation for 'dutch', all other languages will display their footer in English by default. Manual changes can of course be made in 'beamerouterthemelu.sty'.
\documentclass[english, aspectratio=169]{beamer}

% OPTIONS FOR THEME:
% - defaultfooter: if set, change the footer to display Leiden University's default message 'Discover the world at Leiden University' in Dutch or English depending on the set language (above) in the footline.
%   Default: custom footer by theme
%   Option(s): <none>
% - style: if set, can be used to change the title page's author section background colour to the colour of your faculty of choice.
%   Default: midblueLU (#8592BC), like the example PPTX presentation provided by Leiden University
%   Option(s): social, science, humanities, archaeology, fgga, medicine, law
\usetheme[style=science]{lu}

\usepackage{appendixnumberbeamer}
\graphicspath{{graphics/}}

\title{What Makes Leiden University so Great?}
\subtitle{Its People}
\author[surname1, surname2, surname3]{\texorpdfstring{%
    \begin{columns}%
        \begin{column}{0.51\paperwidth}%
            \begin{beamercolorbox}[center, sep=1ex]{author}%
                \leftskip=0.15\paperwidth%
                \usebeamerfont{author}T1.Surname1\ {\footnotesize(1234567)\, ---\\ \href{mailto:T.Surname1@leidenuniv.nl}{T.Surname1@leidenuniv.nl}}%
            \end{beamercolorbox}%
        \end{column}%
        \hskip-0.5pt
        \begin{column}{0.5\paperwidth}%
            \begin{beamercolorbox}[center, sep=1ex]{author}%
                \usebeamerfont{author}T.Surname2\ {\footnotesize(1234568)\, ---\\ \href{mailto:T.Surname2@leidenuniv.nl}{T.Surname2@leidenuniv.nl}}%
                \rightskip=0.15\paperwidth%
            \end{beamercolorbox}%
        \end{column}%
    \end{columns}%
    \nointerlineskip%
    \vspace{-1pt}%
    \begin{columns}%
    \begin{column}{1\paperwidth}%
        \begin{beamercolorbox}[center, sep=1ex]{author}%
            \usebeamerfont{author}T3.Surname3\ {\footnotesize(1234569)\, ---\\ \href{mailto:T.Surname3@leidenuniv.nl}{T.Surname3@leidenuniv.nl}}%
        \end{beamercolorbox}%
    \end{column}%
\end{columns}%
}{T.Surname1, T.Surname2, T.Surname3}}
\institute[LU]{Leiden University}
\date[\today]{Occassion, \today}
\titlegraphic{\includegraphics[height=1.5cm]{logo-universiteitleiden-english.pdf}}


\begin{document}

    \begin{frame}[b]
        \titlepage
    \end{frame}

    \begin{frame}{Some Block and Styling Examples}

        \begin{definition}
           \textbf{ Leiden university} is the \emph{oldest} institution of higher education in the Netherlands.
        \end{definition}
        \begin{alertblock}{Housing}
            If you consider studying at Leiden University and come from far or abroad, start searching for a place to stay \alert{as soon as possible}!
        \end{alertblock}
        \begin{exampleblock}{Open}
            Everyone is welcome!
        \end{exampleblock}
    \end{frame}

    \begin{frame}{Itemise Example}
        \begin{itemize}
            \item This is the first item.
            \item This is the second item.
            \begin{itemize}
                \item This is the first subitem.
            \end{itemize}
        \end{itemize}
    \end{frame}

    \begin{frame}[b]
        \setbeamercolor{closure}{bg=white, fg=white}
            \begin{tikzpicture}[remember picture, overlay]
                \fill[darkblueLU] (current page.north west) rectangle (current page.east);
            \end{tikzpicture}
        	\begin{beamercolorbox}[wd=1\paperwidth,center, sep=1ex, rightskip=0.15\paperwidth]{title}
    		  \usebeamerfont{title}\leftskip=0.15\paperwidth \noindent Questions?\par
            \vspace{0.3\paperheight}
        \end{beamercolorbox}
        \begin{beamercolorbox}[wd=\paperwidth,left,leftskip=0.03\paperwidth]{closure}
            \includegraphics[height=0.3\paperheight]{graphics/logo-universiteitleiden-english.pdf}
        \end{beamercolorbox}
    \end{frame}

    \appendix

    \begin{frame}[allowframebreaks]{References}
        \printbibliography
    \end{frame}

\end{document}
